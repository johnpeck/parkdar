Figure \ref{fig:rxCmdFlow} shows how incoming characters are processed into command strings.
\begin{figure}[ht]
    \begin{center}
        \includegraphics[clip,scale=0.5]{rxCmdFlow}
        \caption{Flow diagram for processing incoming ASCII characters into command strings.\label{fig:rxCmdFlow}}
    \end{center}
\end{figure}

The received character buffer, illustrated in figure \ref{fig:rxBuffer}, is filled by the USART and scanned periodically by the firmware.
\begin{figure}[ht]
    \begin{center}
        \includegraphics[clip,scale=0.7]{rxBuffer}
        \caption{Pointers used to fill, scan, and read from the received character buffer.\label{fig:rxBuffer}}
    \end{center}
\end{figure}

If a \verb8\r8 character is found in the received character buffer, the characters up to that \verb8\r8 become known as a command string.  The command string is copied to the command string buffer, and a state variable is set to let the system know that the buffer is ready to be parsed.
The command string buffer is then processed as shown in figure \ref{fig:cStrFlow} to see if it contains a recognized command.

\begin{figure}[ht]
    \begin{center}
        \includegraphics[clip,scale=0.6]{cStrFlow}
        \caption{Flow diagram for processing the command string buffer.\label{fig:cStrFlow}}
    \end{center}
\end{figure}
